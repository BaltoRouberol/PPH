%---Packages---%
\documentclass[a4paper,11pt]{article}
\usepackage[left=2.5cm,top=2cm,right=2cm,nohead]{geometry}
\usepackage[french]{babel}
\usepackage[T1]{fontenc}
\usepackage[utf8]{inputenc} 
\usepackage{graphicx}
\usepackage{float}
\usepackage{amsmath}
\usepackage{amsfonts}
\usepackage{amssymb}
\usepackage{listings}
\usepackage{mdwlist}
\usepackage[usenames,dvipsnames]{color}
\usepackage[stable]{footmisc}%To include footnotes in 'section' parts
\usepackage{hyperref}
\usepackage{setspace}
\usepackage{eurosym}
\usepackage[section]{algorithm} % [section] is use to define the numbering mode
\usepackage{algorithmic} 
\usepackage{verbatim}
\lstloadlanguages{R}

%---Insertion de code---%
\definecolor{lightgray}{gray}{0.95}

\lstset
{           
backgroundcolor=\color{lightgray},
keywordstyle=\color{Red}\bfseries,
ndkeywordstyle=\color{darkgray}\bfseries,
commentstyle=\color{Green},
stringstyle=\color{Orange},
basicstyle=\scriptsize,       % the size of the fonts that are used for the code
numbers=left,                   % where to put the line-numbers
numberstyle=\tiny,      % the size of the fonts that are used for the line-numbers
stepnumber=2,                   % the step between two line-numbers. If it's 1 each line will be numbered 
numbersep=5pt,                  % how far the line-numbers are from the code
showspaces=false,               % show spaces adding particular underscores
showstringspaces=false,         % underline spaces within strings
showtabs=false,                 % show tabs within strings adding particular underscores
tabsize=2,	                    % sets default tabsize to 2 spaces
captionpos=b,                   % sets the caption-position to bottom
breaklines=true,                % sets automatic line breaking
breakatwhitespace=false,        % sets if automatic breaks should only happen at whitespace
%title=\lstname,                 % show the filename of files included with \lstinputlisting & 
escapeinside={\%*}{*)},         % if you want to add a comment within your code
morekeywords={*,...}            % if you want to add more keywords to the set
extendedchars=true
}

%---Liens---%
\hypersetup{
unicode=false,          % non-Latin characters in Acrobat’s bookmarks
pdftoolbar=true,        % show Acrobat’s toolbar?
pdfmenubar=true,        % show Acrobat’s menu?
pdffitwindow=false,     % window fit to page when opened
pdfstartview={FitH},    % fits the width of the page to the window
pdftitle={PPH - Comment manager une équipe de bénévoles associatifs ?},    % title
pdfauthor={Balthazar Rouberol},     % author
pdfsubject={PPH - Comment manager une équipe de bénévoles associatifs ?},   % subject of the document
pdfcreator={Balthazar Rouberol},   % creator of the document
pdfkeywords={PPH, management, travail d'équipe, bénévolat}, % list of keywords
pdfnewwindow=true,      % links in new window
colorlinks=true,       % false: boxed links; true: colored links
linkcolor=black,          % color of internal links
citecolor=black,        % color of links to bibliography
filecolor=white,      % color of file links
urlcolor= NavyBlue,           % color of external links
bookmarks=true,% show bookmarks bar?
bookmarksopen=false,
bookmarksnumbered = false      
}%

%%---Page de garde---%
\makeatletter
\def\clap#1{\hbox to 0pt{\hss #1\hss}}%
\def\ligne#1{%
\hbox to \hsize{%
\vbox{\centering #1}}}%
\def\haut#1#2#3{%
\hbox to \hsize{%
\rlap{\vtop{\raggedright #1}}%
\hss
\clap{\vtop{\centering #2}}%
\hss
\llap{\vtop{\raggedleft #3}}}}%
\def\bas#1#2#3{%
\hbox to \hsize{%
\rlap{\vbox{\raggedright #1}}%
\hss
\clap{\vbox{\centering #2}}%
\hss
\llap{\vbox{\raggedleft #3}}}}%
\def\maketitle{%
\thispagestyle{empty}\vbox to \vsize{%
\haut{}{\Large \@blurb}{}
\vfill
\vspace{1cm}
\begin{flushleft}

\huge \@title
\end{flushleft}
\par
\hrule height 4pt
\par

\begin{flushright}
\Large \@author
\par
\end{flushright}
\vspace{1cm}
\vfill
\vfill
\bas{}{INSA Lyon - 4BIM\\[0.15cm] \today}{}
}%
\cleardoublepage
}
%\def\date#1{\def\@date{#1}}
\def\author#1{\def\@author{#1}}
\def\title#1{\def\@title{#1}}
\def\location#1{\def\@location{#1}}
\def\blurb#1{\def\@blurb{#1}}
\date{\today}
\makeatother
\title{\textbf{PPH - Comment \og manager \fg{} une équipe de bénévoles associatifs ?}}
\author{par Balthazar Rouberol}
\location{Lyon}
\blurb{%
\begin{center}
	\mbox{\textbf{Projet Personnel en Humanités}}
	\\
	\mbox{\normalsize Tuteur : Joachim Revez}
	\\[3.5cm]
	%\includegraphics[scale = 1]{./Net.jpg}\\[2cm]
\end{center}
\setlength\fboxsep{0pt}
\setlength\fboxrule{1pt}
}% p
